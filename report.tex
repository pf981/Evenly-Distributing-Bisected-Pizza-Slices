\documentclass{article}
\usepackage{amssymb}
\usepackage{amsthm}
\usepackage{amsmath}
\usepackage[hypcap]{caption}
\title{Evenly Distributing Bisected Pizza Slices}
\author{Paul Foster}
\begin{document}
\maketitle

\newtheorem{theorem}{Theorem}
\newtheorem{corollary}{Corollary}[theorem]
\newtheorem{lemma}[theorem]{Lemma}

\section{Introduction}
$N>1$ people are sitting at a table and wish to divide a pizza evenly amoungst themselves. The only way they can cut the pizza is to bisect every slice. After $1$ cut, there are $2$ slices; after $2$ sets of cuts, there are $4$ slices etc. This raises several questions:
\begin{enumerate}
  \item Given $N$ people, is it possible to divide the pizza such that an equal number of slices can be given to each person?
  \item How would you find the least number sets of bisections that are needed to give an equal number of slices to each person?
\end{enumerate}

\section{Analysis}
After $j$ bisections, there will be $2^j$ slices. So we wish to solve the following: given $N \in \mathbb{N}$, find the smallest $j \in \mathbb{N}$ (if it exists) such that $\exists k \in \mathbb{N}$ such that
\begin{equation} \label{eq:solve}
  kN = 2^j
\end{equation}
where $N$ is the number of people, $j$ is the number of bisections and $k$ is the number of slices per person.

\begin{theorem}
$N$ has a solution to~\eqref{eq:solve} iff $\exists l \in \mathbb{N}$ such that $N = 2^l$.
\end{theorem}

\begin{proof}
To prove it by contradiction try and assume that the statemenet is false,
proceed from there and at some point you will arrive to a contradiction.
\end{proof}


\begin{corollary}
If $N$ has a solution to~\eqref{eq:solve}, $j$ is minimised when $j$ = $\log_2 N$
\end{corollary}

\begin{proof}
To prove it by contradiction try and assume that the statemenet is false,
proceed from there and at some point you will arrive to a contradiction.
\end{proof}

\end{document}
